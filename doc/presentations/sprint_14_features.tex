\documentclass{beamer}
\mode<presentation>
{ \usetheme{boxes} }

\usepackage{times}
\usepackage{graphicx}
\usepackage{csquotes}
\usepackage[backend=bibtex]{biblatex}

\title{Dogen v1.0.14, \enquote{Deserto}}

\author{Marco Craveiro \\
  Domain Driven Development
}
\date{\today}

\AtBeginSection[]
{
  \begin{frame}<beamer>
    \frametitle{Outline}
    \tableofcontents[currentsection]
  \end{frame}
}

\bibliography{sprint_14_features}
\begin{document}

\begin{frame}
\titlepage
\end{frame}

\begin{frame}
\frametitle{Language rename}

\begin{itemize}

  \pause

\item The following fields have been renamed:
  \texttt{masd.injection.input\_language} and
  \texttt{masd.extraction.output\_language}.

  \pause

\item New names are \texttt{masd.injection.input\_technical\_space}
  and \texttt{masd.extraction.output\_technical\_space}.

\end{itemize}

\end{frame}

\begin{frame}
\frametitle{Decoration as meta-model elements}

\begin{itemize}

  \pause

\item Originally we used assorted JSON files to store decoration
  data. Each file had its own format, parsing etc.

  \pause

\item Lots of special handling for each different decoration type,
  but ultimately with the same objective: consume them from the model.

  \pause

\item It is now clear that these are one of many examples of entities
  that require modeling at the meta-model level.

\end{itemize}

\end{frame}

\begin{frame}
\frametitle{Decoration as meta-model elements}

\begin{itemize}

  \pause

\item We group these elements as ``decoration''.

  \pause

\item Demonstration on how to use the new decoration elements.

\end{itemize}

\end{frame}

\begin{frame}
\frametitle{Next Sprint}

\begin{itemize}
\item \textbf{Mission statement}: Provide support for profiles at the
  metamodel level.
\end{itemize}

\end{frame}

\end{document}
