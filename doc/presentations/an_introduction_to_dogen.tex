\documentclass{beamer}
\mode<presentation>
{ \usetheme{boxes} }

\usepackage{times}
\usepackage{graphicx}
% \usepackage{biblatex}
% \addbibresourse{an_introduction_to_dogen.bib}

\title{An Introduction to Dogen}
\author{Marco Craveiro \\
    Dept. of Biocomputation \\
    University of Hertfordshire
}
\date{\today}

\AtBeginSection[]
{
  \begin{frame}<beamer>
\frametitle{Outline}
\tableofcontents[currentsection]
\end{frame}
}

\begin{document}

\begin{frame}
\titlepage
v${DOGEN_VERSION}
\end{frame}

\section{Background}

\begin{frame}
\frametitle{What is Dogen}

\begin{itemize}

\item Dogen is a suite of code generation tools designed specifically
  to target source code representations of \emph{domain models}. The
  name is a contraction of ``domain generator''.

\pause

\item The term \emph{domain model} is understood as defined by Evans
  in \cite{Evans04} and summarised by Fowler: ``An object model of the
  domain that incorporates both behaviour and
  data.''\cite{Fowler2015}. \emph{Domain} is the problem domain one
  intends to model (i.e. generate classes to describe).

\pause

\item Most behaviours are problem-specific and thus cannot be
  code-generated. However, there are a few ``basic behaviours'' that
  can be inferred: serialisation, persistence to a store such as a
  relational database, debug printing and so on.

\end{itemize}

\end{frame}

\begin{frame}
\frametitle{Why do we need Dogen?}

\begin{itemize}

\item There are many code generators out there, generating source code
  in many programming languages; so \emph{why do we need yet another
    code generator?}

\pause

\item We have analysed a number of code generators (see the Dogen
  manual for details) and couldn't quite find a suitable one.

\pause

\item Code generators come in two basic flavours: special purpose and
  general purpose.

\pause

\item Special purpose are designed with a single use in mind and users
  are not meant to fiddle with the generated code.

\pause

\item The general purpose code generators we experimented with either
  did not provide extensibility mechanisms at all or were quite
  complex and/or programming language specific. eCore and the Eclipse
  Modeling Framework is an example of this.

\end{itemize}

\end{frame}

\begin{frame}
\frametitle{Properties of an ideal code generator}

\begin{itemize}

\item \textbf{Readability}. Generated code should be very close to human
  generated code; ideally, indistinguishable. A lot of general purpose
  code generators generate unreadable code. However, users gain domain
  insight by looking at it's source representation, so it must be very
  readable.

\pause

\item \textbf{Speed}. In order for the generator to be usable, it must
  be able to generate large amounts of code quickly (less than 10
  secs).

\pause

\item \textbf{Extensibility}. Required in two ways. First, in the
  sense of the ``basic behaviours'' explained above: additional
  serialisation methods, etc. Second, in the sense of domain specific
  behaviours: repetitive patterns in a given domain should be
  code-generatable.

\pause

\item \textbf{Integratability}. Ideally, we should be able to reuse
  existing formats such as eCore, JAXB, etc.

\end{itemize}

\end{frame}

\begin{frame}
\frametitle{Short history of Dogen}

\begin{itemize}

\item Started when taking time off of work around 2012, to create my
  first start-up.

\pause

\item Benefited from the professional experience of several years in a
  code generation team~--- albeit with different technologies: C\# and
  T4 Text Templating.

\pause

\item Originally envisioned as a 6 month project, it has been going
  strong for almost 3 years now and still requires quite a bit of work
  to realise the vision. Work done part-time so progress has not been
  as fast as needed.

\pause

\item Development uses agile methodology, with bi-weekly
  sprints. Emacs org-mode provides a low-ceremony way of keeping track
  of all the work (sprint backlogs, product backlogs, time tracking,
  etc).

\pause

\item Project is very insular and has not yet been exposed to a wider
  community; since the core vision is not yet fully implemented is
  difficult to get developers on board.

\end{itemize}

\end{frame}

\begin{frame}
\frametitle{Where are we now?}

\begin{itemize}

\item Dogen is now a fairly capable C 11 code generator. It is used
  to generate itself, with 15 models in total (over 400 classes), with
  the largest model composed of over 150 classes. It has over 34 test
  models, and almost 800 unit tests.

\pause

\item It supports several basic behaviours (called facets): boost
  serialisation (XML, binary, text), hashing, relational database
  support (via ODB), dumping objects to log files as JSON and provides
  test data generation.

\pause

\item We are currently in the process of moving over to a new text
  templating engine, which will make extensibility much easier.

\end{itemize}

\end{frame}

\section{Internals}

\begin{frame}
\frametitle{What is Dogen}

\begin{itemize}

\item Started when taking time off of work around 2012, to create my
  first start-up.

\end{itemize}

\end{frame}

% \printbibliography

\end{document}
