\documentclass{beamer}
\mode<presentation>
{ \usetheme{boxes} }

\usepackage{times}
\usepackage{graphicx}
\usepackage[backend=bibtex]{biblatex}

\title{An Introduction to Dogen}
\author{Marco Craveiro \\
  Domain Driven Development
}
\date{\today}

\AtBeginSection[]
{
  \begin{frame}<beamer>
    \frametitle{Outline}
    \tableofcontents[currentsection]
  \end{frame}
}

\bibliography{an_introduction_to_dogen}

\begin{document}

\begin{frame}
\titlepage
v${DOGEN_VERSION}
\end{frame}

\section{Background}

\begin{frame}
\frametitle{What is Dogen}

\begin{itemize}

\item Dogen is a suite of code generation tools designed specifically
  to target source code representations of \emph{domain models}. The
  name is a contraction of ``domain generator''.

\pause

\item The term \emph{domain model} is understood as defined by Evans
  \cite{Evans2004} and summarised by Fowler: ``An object model of the
  domain that incorporates both behaviour and data''
  \cite{Fowler2015}. \emph{Domain} is the problem domain one intends
  to model (i.e. generate classes to describe).

\pause

\item Most behaviours are problem-specific and thus cannot be
  code-generated. However, there are a few ``basic behaviours'' that
  can be inferred: serialisation, persistence to a store such as a
  relational database, debug printing and so on.

\end{itemize}

\end{frame}

\begin{frame}
\frametitle{Why do we need Dogen?}

\begin{itemize}

\item There are many code generators out there, generating source code
  in many programming languages; so \emph{why do we need yet another
    code generator?}

\pause

\item We have analysed a number of code generators (see the Dogen
  manual for details) and couldn't quite find a suitable one.

\pause

\item Code generators come in two basic flavours: special purpose and
  general purpose.

\pause

\item Special purpose are designed with a single use in mind and users
  are not meant to fiddle with the generated code.

\pause

\item The general purpose code generators we experimented with either
  did not provide extensibility mechanisms at all or were quite
  complex and/or programming language specific. eCore and the Eclipse
  Modeling Framework is an example of this.

\end{itemize}

\end{frame}

\begin{frame}
\frametitle{Properties of an ideal code generator}

\begin{itemize}

\item \textbf{Readability}. Generated code should be very close to
  human generated code~--- ideally, indistinguishable. Users gain
  domain insight by looking at the source code, so it must be very
  readable.

\pause

\item \textbf{Speed}. In order for the generator to be usable, it must
  be able to generate large amounts of code quickly (less than 10
  secs).

\pause

\item \textbf{Extensibility}. First, in the sense of the ``basic
  behaviours'' explained above: additional serialisation methods,
  etc. Second, in the sense of domain specific behaviours: repetitive
  patterns in a given domain should be code-generatable. Finally, it
  must support multiple programming languages.

\pause

\item \textbf{Integratability}. Ideally, we should be able to reuse
  existing formats such as eCore, JAXB, etc.

\pause

\item \textbf{Merging support}. It should be possible to combine
  hand-crafted code with code-generated code seamlessly.

\end{itemize}

\end{frame}

\begin{frame}
\frametitle{Short history of Dogen}

\begin{itemize}

\item Started when taking time off of work around 2012, to create my
  first start-up.

\pause

\item Benefited from the professional experience of several years in a
  code generation team~--- albeit with different technologies: C\# and
  T4 Text Templating.

\pause

\item Originally envisioned as a 6 month project, it has been going
  strong for almost 3 years now and still requires quite a bit of work
  to realise the vision. Work done part-time so progress has not been
  as fast as needed.

\pause

\item Development uses agile methodology, with bi-weekly
  sprints. Emacs org-mode provides a low-ceremony way of keeping track
  of all the work (sprint backlogs, product backlogs, time tracking,
  etc).

\pause

\item Project is very insular and has not yet been exposed to a wider
  community; since the core vision is not yet fully implemented is
  difficult to get developers on board.

\end{itemize}

\end{frame}

\begin{frame}
\frametitle{Where are we now?}

\begin{itemize}

\item Dogen is now a fairly capable C++ 11 code generator. It is used
  to generate itself, with 15 models in total (over 400 classes), with
  the largest model composed of over 150 classes. It has over 34 test
  models, and almost 800 unit tests.

\pause

\item It supports several basic behaviours (called facets): boost
  serialisation (XML, binary, text), hashing, relational database
  support (via ODB), dumping objects to log files as JSON and provides
  test data generation.

\pause

\item We are currently in the process of moving over to a new text
  templating engine, which will make extensibility much easier.

\pause

\item I will be using Dogen in my research work.

\end{itemize}

\end{frame}

\section{Internals}

\begin{frame}
\frametitle{Tools in the suite}

\begin{itemize}

\item There are two main tools in the Dogen toolbox: \emph{knitter}
  and \emph{stitcher}. Knitter is the code generator. Stitcher is a
  text-templating engine.

\pause

\item \emph{knit} is the library that implements all of the knitting
  logic. \emph{stitch} is the library that implements all of the
  stitching logic.

\end{itemize}

\end{frame}

\begin{frame}
\frametitle{Knit Architecture: Frontends}

\begin{itemize}

\item Follows the traditional approach of compilers, with a split
  between frontends, middle-end and backends.

\pause

\item At present we have two frontends: Dia XML and JSON.

\pause

\item Dia XML is the file format used by the Dia diagramming
  application. It can be used to generate UML diagrams. These UML
  diagrams can be augmented with a few Dogen specific annotations and
  these help enrich the generated code.

\pause

\item JSON frontend is not complete yet, but it is used to expose
  ``system libraries'' to the code generator such as Boost, STL,
  etc. In the future it will be a fully-formed front-end.

\pause

\item Other frontends will be introduced over time such as XSD (with
  JAXB annotations), etc.

\end{itemize}

\end{frame}

\begin{frame}
\frametitle{Knit Architecture: Middle-end}

\begin{itemize}

\item The middle-end is a model we called SML - Simplified Modeling
  Language. It is a simple meta-model that is rich enough to store the
  details of the object model in a programming language neutral
  format.

\pause

\item SML has a number of internal transformations that setup the data
  structures in a shape that is easy to query for code generation
  purposes.

\pause

\item Simpler languages can generate code directly out of SML. More
  complex languages such as C++ require further processing.

\end{itemize}

\end{frame}

\begin{frame}
\frametitle{Knit Architecture: Backends}

\begin{itemize}

\item At present there is only one backend: C++-11. We wanted to
  ensure we had a decent enough architecture overall before we started
  to add more backends.

\pause

\item The C++ backend has is own data model, obtained as a
  transformation of SML. This ensures the data is as close as possible
  to the code-generation requirements and minimises the complexity of
  the outputting code.

\pause

\item Once this transformation is done, we pass the objects to the
  \emph{formatters}~--- the classes responsible for code generation.

\pause

\item Formatters were originally written as simple ``print
  statements''. Predictably, this approach proved cumbersome; a way of
  templating them was required. We had experience with T4 so we knew
  what we were looking for, but could not find anything suitable. So
  we created stitch.

\end{itemize}

\end{frame}

\begin{frame}
\frametitle{Stitch}

\begin{itemize}

\item Stitch is very simple: the user generates a text file according
  to a T4-like grammar, and stitch generates a C++ file with ``print
  statements'' that outputs instances of that template.

\pause

\item Using templates means that one can start with a real source code
  file and ``templatise'' the parts of it that are
  instance-dependent. It makes extensibility much easier.

\pause

\item In a template, users can freely mix C++ code with text. It
  follows exactly the same rules as T4. Stitch can be thought of as a
  subset of T4.

\pause

\item Most knit formatters are implemented as Stitch templates, but
  the transition has not been fully achieved.

\end{itemize}

\end{frame}

\begin{frame}
\frametitle{Demo}

\begin{itemize}

\item Editing a diagram in Dia. JSON input files.

\pause

\item Quick tour of the generated code, covering the different
  facets.

\pause

\item Making a modification, running knitter and looking at the diffs
  in git.

\pause

\item Quick run-through a stitch template. Change it, run knitter and
  look at the diffs in git.

\end{itemize}

\end{frame}

\begin{frame}
  \printbibliography
\end{frame}

\end{document}
